\documentclass[12pt]{article}
\usepackage[utf8]{inputenc}
\usepackage[T1]{fontenc}
\usepackage{amsmath,amsfonts,amssymb}
\usepackage{graphicx}
\usepackage{a4wide}
\title{Reconstructed abstract of the paper ``Put the paper title or DOI here''}
%\author{not specified, not necessary here}
\date{}
\begin{document}
\maketitle

\begin{abstract}
We present an approach to graph neural networks based on the concept of diffusion processes. We formalize the interactions between the nodes of the graph through partial differential equations, which allows us to more accurately model the propagation of information through the graph structure. The proposed method, called GRAND (Graph Neural Diffusion), demonstrates a significant improvement in accuracy compared to existing approaches, thereby achieving more efficient feature extraction and prediction on graphs. The experimental results confirm the effectiveness of our approach in a number of tasks related to graph data analysis. 
\end{abstract}
\paragraph{Keywords:} GRAND, graph neural network, partial differential equation, diffusion.

\paragraph{Highlights:}
\begin{enumerate}
\item We describe a broad new class of GNNs based on the discretised diffusion PDE on graphs and study different numerical schemes for their solution.
\item We use an attention mechanism with trainable parameters to simulate the diffusion process and ensure its stability. 
\item The presented architecture of graph neural networks showed results that surpassed popular and advanced representations of graph neural networks. 
\end{enumerate}

\section{Introduction}
This article attracted me with a new approach to neural networks, which have a deep mathematical basis. This solution was taken from a physical process, which is more closely related to the real world. This may be an advanced solution to many problems related to graph neural networks. 
%\begin{figure}
%\includegraphics[scale=0.35]{SVD_derint}
%\caption{A rigorous description of what the reader sees on the plot and the consequences of the shown result}
%\end{figure}

\bibliographystyle{unsrt}
\bibliography{Name-theArt}
\end{document}